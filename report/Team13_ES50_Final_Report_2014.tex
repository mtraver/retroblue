\documentclass{es50report}

\usepackage{setspace}

\usepackage{palatino}


\title{RetroBlue}
\projDescription{A Bluetooth-enabled vintage rotary phone that can be paired with a cell phone to make and receive calls.}
\authors{Kyle Solan, Michael Tingley, and Michael Traver}

\begin{document}
    \maketitlepage

    \begin{abstract}
        RetroBlue is a modern cell phone accessory with an old-timey feel. We took a vintage rotary phone, an Arduino board, and a Bluetooth chip and combined them to create RetroBlue, a fully functional Bluetooth accessory that can make, dial, and receive calls via a cell phone. A user simply pairs his or her cell phone with RetroBlue and all calls are then forwarded to the unit. We worked extensively to convert the antiquated circuitry within the phone to portable battery-powered components and developed custom code and circuitry to pass calls to and from the rotary phone's hardware.
    \end{abstract}
    \newpage
    \doublespacing

    \section{Introduction}
    TODO

    \section{Design}
        We had clear goals and objectives that guided much of the design process for our project. We wanted to emulate the experience of using a vintage rotary phone, but rotary phones interface poorly with modern landlines, and furthermore, landlines are becoming less and less common altogether. A rotary phone that operates as a cell phone was an early idea, but it seemed inconvenient to have a dedicated "cell phone" that was actually a rotary desktop phone. We thought this would be better suited as a desktop accessory that could pair with a cellphone.
        
        The Bluetooth Hands-Free Profile (http://en.wikipedia.org/wiki/Bluetooth\_profile\#Hands-Free\_Profile\_.28HFP.29) is a set of standards and protocols commonly used in car Bluetooth accessories that allow an external device to handle a cell phone's dialing, calling, and receiving operations. We thought we could use these to create our "desktop accessory" and build a free-standing vintage phone that pairs easily with cell phones. We found a number of Bluetooth chips online that supported HFP and provided a serial interface for communication.
        
        Traditional landline phones are made up of very simple hardware. The hardware itself is merely responsible for transmitting and outputting signals sent via the landline, and all logic and signal interpretation is carried out by the service provider. Our system, in contrast, needed to be aware of its own state, interpret its own signals, and transmit all this information to the Bluetooth module via digital serial communication. For this we elected to use an Arduino board, which provided the ability to interface with our diverse set of components and design our own event-handling logic.

        With this basic system in mind, we began our implementation process. Initially, we had the idea to use our Arduino as a simulated service provider and connect it directly to the phone's external data/power line. With this, we would not have needed to modify the phone's internal hardware. This situation, however, quickly proved impractical. Landline phones are meant to run on high voltage (50-90V), and much of the hardware is activated via large voltage fluctuations. This was poorly suited to our goal to have a free-standing portable accessory and would have required a lot of external circuitry to meet these power requirements. Instead, we found we could modernize the phone circuitry to run on a much lower voltage well-suited to battery power (3.3-5V) if we worked directly with independent components within the phone. 

        However, the electromagnetic motor that drove the ringer still ran on higher voltage even when isolated. We decided to remove this motor and replace it with a small solenoid that ran on 12-18V, suitable for battery power. It produced a good ring with a fraction of the power.

        With this initial hurdle overcome, we began by parsing the project up into independent circuits and components: the dialer, the ringer, the earpiece, the microphone, and the hook. We worked to implement circuits and test functionality of each of these components independently. Afteward, we connected each of them to the Arduino/Bluetooth module and wrote Arduino code to oversee the phone's global state and handle signals appropriately. After we achieved proper overall functionality, we consolidated these independent circuits into a single circuit operating on a single power source (except for the ringer solenoid, which required a bit more power).

    \section{Parts List}
        \begin{description}
            \item[Rotary phone.] A standard functional rotary telephone. As far as we can find, these are no longer manufactured, and most modern ``vintage'' telephones have fake rotary dials with touch-tone keys. We want the real thing, so we'll likely have to resort to eBay (\url{http://www.ebay.com/sch/i.html?_trksid=p2050601.m570.l1313.TR12.TRC2.A0.H0.Xrotary+phone&_nkw=rotary+phone&_sacat=0&_from=R40}). Estimated price is \$20-40 with shipping.
            \item[Bluetooth audio module.] An integrated circuit with Bluetooth transmitter and simple firmware. We intend to use the chipset to interface with a cell phone, with the ability to receive and dial/make calls. This means we need a chipset that supports the hands-free profile (HFP). We have found a chipset that meets these requirements and includes a very comprehensive user manual detailing operation and the chipset API (\url{http://www.digikey.com/product-detail/en/RN52-I%2FRM/RN52-I%2FRM-ND/3884028}). Price is \$23.30 before shipping.
            \item[LCD module.] A small display for displaying caller ID information to users. We think 24 alphanumeric characters should be sufficient, and the following display will be adequate: \url{http://www.digikey.com/product-detail/en/NHD-0212WH-AYGH-JT%23/NHD-0212WH-AYGH-JT%23-ND/1701138}. Price is \$9.70 before shipping.
            \item[Arduino board.] Needed as the logic center for our system. We will write a substantial amount of code to power the Arduino and use it to process data from the Bluetooth module and rotary phone and output it to the LCD/Bluetooth module.
            \item[Misc. electrical components.] Wires, solder, op-amps, etc.
        \end{description}

    \section{Project Implementation}
    TODO

    \section{Team Management}
        We all worked collaboratively on most parts of the project. We all came into the project with similar background; all of us are computer science majors that are generally comfortable with coding but not very comfortable with the hardware side of the project. Nonetheless, there were a few areas of specialization that each member took responsibility for.

        \begin{description}
            \item[Dial pulses and handset]
                Michael Traver was responsible for a lot of the hardware wiring. He worked on figuring out where each wire should be attached in order to sense a dial pulse or to sense the state of the handset (on the hook/off the hook). Michael Tingley improved the code for pulse sensing. Instead of sensing each pulse individually, he came up with the algorithm to listen for the `length' of the input to determine which number was dialed, which proved to be more accurate.
            \item[The ringer]
                Michael Tingley was responsible for much of the work with the ringer. He was responsible for early prototyping of a ringer and for designing the relay circuit to fire the solenoid. Kyle Solan was responsible for the engineering that went into the final ringer. Kyle modified the original ringer and mounted the new one, and configured the solenoid and tweaked the code in order to produce a sound that could be convincing for a 1940s-style ring.
            \item[Microphone and speaker]
                Kyle and Traver did most of the work on the microphone and ringer. Traver worked diligently to reduce echo and increase volume of the microphone. Kyle was responsible for speaker output. Both worked on figuring out how mic biasing works (e.g., what configuration of capacitors to use), and how to correctly ground the circuit. (We had some issues with grounding the circuit because both the speaker and the microphone used the same cable for grounding.)
            \item[Bluetooth module]
                Everyone on the team worked on the bluetooth module. This was easily the hardest part of the project. We had issues soldering, using the correct voltages and baud rates, providing the correct power, and reading the output using the correct encoding. It's difficult to break down each challenge by who solved it, since we all pooled our brainpower when trying to decipher cryptic tutorials online or debugging the circuits with the oscilloscope and multimeter.
            \item[Changing voltages]
                Throughout working with the bluetooth module, we found out that it only accepted/output 3.3V signals. Since the Arduino worked with 5V signals, this turned out to be a problem. Tingley and Traver worked on rigging up circuits to fix this. To convert 5V to 3.3V, we were able to use a simple voltage divider. However, to convert 3.3V to 5V, we had to use a more complex setup of transistors and a 5V input bias. (In the end, the Arduino turned out to be sensitive enough to 3.3V as opposed to 5V, and so we were able to exclude this circuit in the final project.)
            \item[Circuit reduction, portability]
                Traver and Kyle worked on making the circuit smaller and more portable. This involved condensing everything from 3 long circuit boards to a single short one. In addition, they worked on trimming wires as appropriate and carefully compacting everything inside of the phone case so that the device could be made portable. All three of us worked on wiring up the circuit with batteries and making it portable!
        \end{description}

    \section{Outlook and Possible Improvements}
        \begin{description}
            \setlength{\parskip}{2mm}
            \item[LCD display]
                We accomplished all of our major objectives and finished most of our reach goals. The one main thing that we would do with additional time is to use an LCD display to display useful information for the user.

                One of the things that we would like to do is to use  the LCD to display numbers that the user has dialed so far during a dialing sequence. We found that it is very disorienting to dial a number on an old rotary phone. Since you have to wait several seconds between entering numbers, it is very easy to forget where you were. Furthermore, since we read numbers using analog circuitry, it would be useful to get visual feedback that the number was interpreted correctly.

                Additionally, we would like to be able to use the LCD to display caller information. It would be relatively easy to display the incoming number. If possible, we would also like to use the LCD to display the caller's name, using the phonebook in the bluetooth-linked phone. However, this appears to be very difficult, and would require custom phone querying using the data pipe mode on the bluetooth module.
            \item[Texting mode]
                Assuming we care nothing at all about usability, we would like to include a texting mode. This would be virtually unusable, but the basic idea is that you lift the handset up and wait 10 seconds until you hear a beep from the phone. Then you're in texting mode. You can enter letters using a T9-style input method (if we have the LCD, that can show letters already entered). To enter a letter, you simply wait a number of seconds between entries and the letter will automatically be entered. To send the text, you would put the handset down, and then dial the number to send the text to.

                If we have the LCD, we could even display received text messages on it. Because why not.
        \end{description}

    \section{Acknowledgments}
    TODO

    \section{Disclaimer}
        It is okay with us to share our report, codes that we wrote, and photos and videos of our product.

    \section{References}
    TODO

    \appendix
    \section{Appendix}
    TODO

\end{document}
